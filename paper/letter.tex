\documentclass[a4paper, usenatbib, 12pt]{article}
\usepackage{subfig}
\usepackage{float}
\usepackage{wrapfig}
\usepackage{graphicx}
\usepackage{amsmath}
\usepackage{amssymb}
\usepackage{booktabs}
\usepackage{cite}
\usepackage[icelandic,spanish,english]{babel}
\usepackage[T1]{fontenc}
\usepackage[utf8]{inputenc}
\usepackage[top=3.5cm, bottom=2.5cm, left=3.5cm,right=3.5cm]{geometry} 

%----------------------New commands -------------------

\newcommand{\tol}{Tololo 1214-277}
\newcommand{\lya}{Ly$\alpha$}
\newcommand{\hb}{H$\beta$}
\newcommand{\ha}{H$\alpha$}
\newcommand{\oiii}{[OIII]}
\newcommand{\oii}{[OII]}
\newcommand{\nii}{[NII]}
\newcommand{\esca}{erg cm$^{-2}$ s$^{-1}$ \AA$^{-1}$}
\newcommand{\esc}{erg cm$^{-2}$ s$^{-1}$}
\newcommand{\es}{erg s$^{-1}$}
\newcommand{\esa}{erg s$^{-1}$}
\newcommand{\kms}{km s$^{-1}$}
\newcommand{\apj}{ApJ}  
\newcommand{\jcap}{JCAP}  
\newcommand{\apjs}{ApJS}  
\newcommand{\apjl}{ApJL}  
\newcommand{\aj}{AJ}  
\newcommand{\mnras}{MNRAS}  
\newcommand{\mnrassub}{MNRAS accepted}  
\newcommand{\aap}{A\&A}  
\newcommand{\aaps}{A\&AS}  
\newcommand{\araa}{ARA\&A}  
\newcommand{\nat}{Nature}  
\newcommand{\physrep}{PhR}
\newcommand{\pasp}{PASP}    
\newcommand{\pasj}{PASJ}    
\def\simgt{\lower.5ex\hbox{\gtsima}}
\def\simlt{\lower.5ex\hbox{\ltsima}}
%------------------------------------------------------

\begin{document}
\pagestyle{empty}
\noindent
\textbf{Þú ert jörðin}
\\
\\
JEFR$^{1}$, MCRG$^1$, JNGC$^2$, MG$^3$, MD$^3$
\\
\\
\scriptsize
{$^1$ Bogota
\\
$^2$ Tucson
\\
$^3$ Oslo
\normalsize
\\
\\
\textbf{
  Star-forming Compact Dwarf Galaxies (CDGs) resemble the expected
  pristine conditions of the first galaxies in the Universe.    
Before the observational detection of the first galaxies becomes
reality, CDGs are the best systems to test our ideas on primordial
galaxy formation and evolution.    
Here we report on one of such CDGs, \tol, which presents
a broad symmetric Lyman-$\alpha$ emission that had evaded theoretical
interpretation so far. 
We explain these features by two different models: an
homogeneous sphere undergoing gas bulk rotation and an interstellar
medium composed by clumps with random motions.
It is the first time that an observed \lya\ spectrum can be explained
assuming these physical conditions.
We  find that both models independently require high velocities that
translate into a dynamical mass at least $10$ times larger than the
neutral mass hydrogen in the galaxy. 
We argue that a possibility to explain the high
dynamical mass is the presence of a super-massive black hole. 
Ionization emission lines present in this galaxy support the idea that
\tol\ might have harbored an an Active Galactic Nucleus  
The implications of our findings for the study of LAEs, including the
first generation of galaxies are manifold. 
It demonstrates the importance of considering rotation and multiphase
physics under the possible conditions shaping the \lya\ spectra of the
first galaxies. 
Additionally, if future kinematic maps of \tol\ confirm the high
velocities postulated in our model, it would provide new
evidence for dwarf galaxies as hosts of active supermassive black
holes.  
}  



The first generation of galaxies trace our cosmic origins. 
They were the first steps in the evolution of galaxies such as the Milky
Way. 
In the standard Big Bang cosmology the only chemical elements that
were created in the nucleosynthesis process were Hydrogen, Helium and
Lithium.  
Heavier elements must have been created in stellar evolution process. 
Therefore, we expect the first generation of
galaxies to be metal free and rich in Hydrogen. 
This kind of primordial galaxies have not been detected yet. 
However, dwarf star forming galaxies with a low metallicity content
are seen as templates to understand the early galaxy evolution process. 

Almost fifty years ago \cite{PartridgePeebles} it was realized that
young galaxies could be detected through a strong Lyman-$\alpha$ line
emission.  
This theoretical prediction was only confirmed thirty year later on
distant, relatively young, not primordial, galaxies.
Currently Lyman Alpha Emitting (LAE) galaxies are commonly targeted
in surveys. 
The presence of the Ly-$\alpha$ emission line provides confirmation of
the distance of a galaxy while provides clues about the stellar
population and inter-stellar medium conditions regulating the
Ly-$\alpha$ emission. 

The Ly-$\alpha$ emission line is not exclusive of distant galaxies. 
Any galaxy with low dust content and ongoing star formation has the
right conditions to show this line.  
There are, for instance,  local Universe surveys that target
Ly-$\alpha$ emission in nearby dwarf star forming galaxies 
\cite{LARS}. 
The study of nearby LAE samples has allowed the study of other
indicators that might be more difficult to obtain for distant galaxies
such as morphology, dust attenuation, neutral hydrogen contents and
ionization state.  

However, the physical interpretation of Ly-$\alpha$ observations is
not straightforward. 
This is due to the resonant nature of the Ly-$\alpha$ line. 
A Ly-$\alpha$ photon follows a diffusion-like process before escaping
the galaxy or being absorbed by dust. 
The resulting line profile becomes sensitive to the dynamical, chemical
and thermal conditions in the interstellar medium. 
There are very few analytically tools available to interpret the
Ly-$\alpha$ line.
They are applicable only in very few cases of highly symmetrical
conditions, which are hardly met in real astrophysical systems.
For these reasons the interpretation of Ly-$\alpha$ observations
require state-of-the-art Monte Carlo radiative transfer simulations.   
Recent advances in these computational models have explored the
effects that bulk rotation and a multiphase state in the interstellar
medium should have on on the Lyman-$\alpha$ line.

The most important consequence of bulk rotation is that, even for a
spherical gas distribution, the line morphology now depends on the
viewing angle respect to the rotation axis.  
For a line of sight perpendicular to the rotation axis the intensity
and the line center and the line width increase with rotational
velocity. 
When the rotational velocity is close to the half-line width of the
static line the line becomes single peaked and symmetric, a unique
feature that other theoretical models find difficult to reproduce
without introducing more complexity into the gas distribution. 

This is the case of a multiphase interstellar medium.
The model is more complex as it must take into account the behavior
of neutral dense hydrogen clumps dispersed in a tenuous intraclump medium.  
Corresponding to the elaborate physical conditions in the model,
there are a wide variety of possible emerging spectra.
This variety includes single, double and triple peaked
spectra with different degrees of symmetry around the line's center.


\tol\ is a compact star forming dwarf galaxy that presents a
strong Ly-$\alpha$ emission \cite{Thuan97} with two puzzling 
features: it is symmetric and single peaked.
Commonly, the Ly-$\alpha$ line has a single or asymmetric double peak. 
These two special features in \tol\ had evaded a physical
interpretation so far, because bulk rotation and a multiphase ISM can
independently explain the \lya\ line in \tol. 

Figure 1. summarizes our findings.
Dots represent the observational data for \tol with the
overplot from our best fit models from the full radiative transfer
simulations for a rotating sphere (continuous line) and a clumpy ISM
(dashed line).

The best parameters in the rotation model are a rotational velocity of 
$v_{max}=300^{+xx}_{-xx}$\kms, an optical depth $\log\tau=7^{+xx}_{-xx}$ and a temperature
of $T=1.5^{+xx}_{-xx}\times 10^{4}$K. 
This translates into a column density of $\log N_{HI} /
\mathrm{atoms\ cm}^{-2} =  20.5^{+xx}_{-xx}$.  
This model is also able to constrain the angle between the rotation
axis and the observational line-of-sight to
$\theta=65^{+xx}_{-xx}$$^{\circ}$.  

In the multiphase model the best constrained parameters are
the clump velocity dispersion  $\sigma_{\rm{cl}}=71^{+17}_{-25}$ \kms,
the clumps outflowing velocity $v_{\infty, \rm{cl}}=79^{+167}_{-60}$
\kms and the fraction of the \lya emission that is  coming
from the cold clumps  
$P_{cl}=0.72^{+0.20}_{-0.32}$. 
The multiphase model assumes that the clouds are distributed over a
sphere of $5$kpc in radius, close to the $\approx 4$kpc physical size
of \tol\ as determined by optical imaging.
The assumed physical size and the velocity dispersion $\sigma_{\rm
  cl}$ correspond to a  dynamical mass of  $3.0^{+1.5}_{-1.8}\times
10^9$ M$_{\odot}$. 

Radio surveys of the 21cm line have put an upper limit to the neutral
hydrogen mass in \tol\ of $M<2.65\times 10^{8}$ M$_{\odot}$ \cite{pustilnikmartin07}. 
In the case of the rotation model, this information help us to constrain the
diameter of the HI region where the \lya\ emission and transfer takes
place. 
We find this size to be in the range $0.11 < D/\mathrm{kpc}<0.34$, one
order of magnitude smaller than \tol's size in the optical.   
That size and the rotational velocity of $300$\kms put a constraint on
the dynamical mass of $4.5^{+2.1}_{-2.4} \times 10^{9}$M$_{\odot}$.

This is evidence for a dynamical mass at least 11 (multiphase case) to
17 (rotation case) times higher than the mass in neutral hydrogen.  
There are three main possibilities to explain the dynamical 
mass: stars, dark matter and a supermassive black hole. 
The stellar mass in old stars has been constrained by non-detections
in the K-band to be $<XXX$M$_{\odot}$.
Considering a standard dark matter halo, it can only contribute to
$xx$ M$_{\odot}$ in the region of interest. 
A super massive black hole remains an open possibility. 
Recent observations of ultra-compact dwarf galaxies
\cite{Seth2014} have confirmed the presence of super-massive black
holes containing $15\%$ of the total object mass, suggesting that
there is a large population of undetected black holes in dwarf
galaxies.  


A future observational test to clarify the physical nature of
\tol\ would require integral field unit measurements spatially
resolving its spatial extent. 
\tol\ spans a region of $4$ arcseconds,
an instrument such as the Multi Unit Spectroscopic Explorer with its
nominal $0.2$ arcseconds spatial sampling over a $1.0$ arcminute field
in wide-field mode could provide a coarse mapping of different
ionization lines to infer a kinematic map.
Another observational test includes the measurement of the escape
fraction of Ly continuum ionizing radiation. 
In the rotational model this fraction should be zero, while
the multiphase model predicts that in this case it should be around
$0.5^{+1.0}_{-0.4}$\%. 

All in all, the mere existence of a strong LAE galaxy with a broad,
symmetric line is interesting.
It raises the question whether some high redshift LAEs have asymmetric
lines because the blue half was truncated by the intergalactic medium.
In this case the \lya\ radiation could emerge as a low surface
brightness glow, which may be connected to \lya\ halos, while also
influencing the way LAEs can be used as a probe of reionization. 

These findings demonstrate the importance of including rotation and multiphase
conditions as features to model the \lya\ line in high redshift
galaxies.
Additionally, if the hypothesis of a supermassive black
hole in \tol\ proves to be consistent with future observational
kinematic maps, it could correspond to a so far undetected black hole
in a dwarf galaxy, providing a new way to test and probe
theories on the co-evolution of galaxies and black holes in the first
generation of galaxies.  

\begin{figure}
\begin{center}
\includegraphics[width=0.8\textwidth]{CLARA-TOL-main.pdf}
\caption{{\bf Broadn, single peaked and symmetric Ly-$\alpha$ emission of \tol.}
  Dots correspond to the observational data. The line shows the results
of our best model from a full radiative transfer simulation both for the rotation and multiphase models.}
\end{center}
\end{figure}

\bibliography{references}{}
\bibliographystyle{plain}

\newpage 

\section*{\tol\ characteristics}


\begin{table}
\begin{center}
\begin{tabular}{lc}
$\alpha$(2000)$^{a}$ & 12h17min17.1s\\
$\delta$(2000)$^{b}$ & -28d02m32s\\
$l$, $b$ (deg) & 294, 34\\
$m_V$ & 17.5\\
  M$_V$ & -17.6\\ 
$v$(km s$^{-1}$) & 7795\\
Ly-$\alpha$ (erg cm$^{-2}$ s$^{-1}$ \AA$^{-1}$)& $8.1\times 10^{-14}$ \\
Ly-$\alpha$ EW & $70$\AA\\
H$\beta$ (erg cm$^{-2}$ s$^{-1}$ \AA$^{-1}$) & $1.62\times 10^{-14}$ \\
$21$cm (Jy km s$^{-1}$)& $<0.10$ \\
\end{tabular}
\end{center}
\caption{Basic observational characteristics of TOL1214-277
  \cite{Thuan97}\\} 
\end{table}

FIXME: Add non detection in 2MASS %\url{http://vizier.u-strasbg.fr/cgi-bin/VizieR?-source=B/2mass}

\tol receding velocity is $7785\pm 50$km s$^{-1}$, which translates
into a distance of $106.6$ Mpc (Hubble constant 73 Mpc km$^{-1}$
s$^{1}$)
Its metallicity is $\sim Z_{\odot}/24$ \cite{Izotov04} as derived from optical
spectroscopy. 


The observed flux for the Lyman alpha line is $\sim
8.1\times 10^{-14}$ erg cm$^{-2}$ s$^{-1}$ \cite{Thuan97}
and a Equivalent Width of $70$\AA and its H$\beta$ flux is 
$1.62\times 10^{-14}$ erg cm$^{-2}$ s$^{-1}$ \AA${-1}$
\cite{Izotov04} which gives a Ly$\alpha$/H$\beta$ flux ratio of
4.9$\pm$0.1. The Ly-$\alpha$ flux values correspond to luminosities of
$L_{Ly\alpha}=2.2\times 10^{42}$ erg s$^{-1}$ over a $20$\AA
bandwidth, which in turns translates  into a star formation rate of
$2.0$ M$_{\odot}$ yr$^{-1}$ using a standard conversion factor between
luminosity and star formation rate of $9.1\times 10^{-43}$
$L_{Ly\alpha}$ M$_{\odot}$ yr$^{-1}$. 
The absolute magnitude in the $V$ band translates into a luminosity of
$8.9\times 10^{8}$ L$_{\odot}$.
% using http://tomdwelly.com/tools_fluxtolum.php
Comparing this ratio with the theoretical expectation from case B
recombination of $23.3$ \cite{Hummer1987} one can estimate an escape
fraction of $20$\% for Ly$\alpha$ radiation.

The optical emission  comes from a   region with approximate diameter
4 kpc \cite{Fricke01}. 

There is an upper limit for the  
integrated flux of $<0.10$ Jy km s$^{-1}$, which translates into a
upper limit for the HI mass of $M<2.65\times 10^{8}$ M$_{\odot}$
\cite{pustilnikmartin07}.  


Interpretation by \cite{mashesse03}.


\section*{The Rotation Model}

The rotation model corresponds to the work presented in
\cite{GaravitoCamargo2014}. 
In that model the Ly-$\alpha$ photons are propagated 
within a spherical and homogeneous cloud of HI gas undergoing solid
body rotation.
The sphere is fully characterized by three parameters: the optical
depth $\tau$ measured from the center to its surface, the HI
temperature $T$, and the linear surface velocity $V_{\rm max}$.  
Photons are emitted at their natural frequency from the center of the
sphere. 
Including the effect of dust only changes the overall line
normalization but not its shape.  
The results we report in the main body of the paper do not include any
dust model.
In the current work, the radiative transfer simulations were done
using Monte Carlo simulations with the code \texttt{CLARA}
\cite{CLARA}.  

The first important effect of rotation is that it breaks the spherical
symmetry of the static case. 
Now the line's observed morphology depends on the angle $\theta$ between the
line-of-sight (LOS) and the rotation axis. 
LOS parallel to the rotation axis tend to observed the line without
any modification from rotation, while the perpendicular LOS will
observe a maximal change in the line's morphology due to rotation.

The main change in the line's morphology is that it broadens and the
intensity at the center increases. 
For high enough rotational velocities the intensity at the peak's
center increases so much that the line goes from double to single
peaked, sometimes slightly triple peaked.
This is the feature that allows this model to fit the observational
features of \tol.

There is a concise analytical description for those features.
This description takes into account how different parts of the
sphere's surface shift in frequency the \lya photons. 
Different shifts in frequency come from different values for the projected
velocity along the LOS. 
As presented in \cite{CLARA}, using the analytical solution for the
\lya sphectra of a static sphere plus the right frequency shifts
computed from geometrical considerations, one is able to produce an
analytical solution for the rotating sphere that reproduces the main
features found using the full numerical simulation.

The analytical solution for the rotation sphere was the base to
perform the Markov Chain Monte Carlo Calculation using the
\texttt{emcee} implementation.
We explore flat priors on $V_{\rm max}$ $\log_{10}\tau$, $\log_{10} T$ and
$\theta$ using $XX$ steps.
The results are summarized in Figure $XX$. 
From this model we find that the fiducial parameters that could
explain the broad features in \tol\ are $V_{\rm max}$, $\log \tau
=XX$, $\log_{10} T = $ and  $\theta =$






\section*{The Multiphase Model} 

The idealized multiphase model consists of spherical, cold, dens
clumps of neutral hydrogen (and dust) embedded in a hot, ionized
medium. 
The clumps also have a random and an outflowing velocity
component which totals the number of parameters describing the model
to be $14$. 
In order to map out this large parameter space, we randomly drew
$2500$ sets of parameters within a observationally realistic range
(based on the considerations of \cite{Laursen2013ApJ...766..124L})
yielding a large variety of single-, double- and triple-peaked
spectra. 
The full analysis of the the spectral features as well as
more details on the radiative transfer are presented in
\cite{Gronke2016}.    
 

For the current work, we computed the $\chi^2$ for each of the $2500$
parameters yielding the best fit parameters of $(...)$. Here, the
subscripts ${}_{\rm cl}$ and ${}_{\rm ICM}$ stand for the quantities
filling the clumps and the medium between the clumps,
respectively. 
Furthermore, we found that some parameters such as the magnitude of
the random clump motion $\sigma_{\rm cl}$ improved the fit
significantly whereas others did not.  

Qualitatively as \tol\ possesses a very wide spectrum which can be
achieved by subsequent scatterings off (relatively) fast moving clumps
while the multi-phase nature (i.e., the existence of low-density
channels) ensures the high flux at line center as observed.     

\section*{Physical Interpretation}

Both the rotation and the multiphase model constrain the typical
velocity $v$ of the HI gas, with and additional constrain on the typical
size for the emission region $r$ on could estimate a dynamical mass with 

\begin{equation}
M_{\rm dyn} = \frac{v^{2}r}{G} = 1.16\times10^{9}
\left(\frac{v}{100\ \mathrm{km\ s}^{-1}}\right)^2\left(\frac{r}{\mathrm{kpc}}\right) M_{\odot}
\end{equation}

From the multiphase model we obtained $v=71_{-25}^{+16}$ \kms
and $r=5$ kpc. 
This corresponds to a dynamical mass of $M_{\rm
  dyn}=5.9^{+3.1}_{-3.4}\times 10^{9}$ $M_{\odot}$ which is at least
$20$ times the HI mass estimated from observations.  

In the rotation model the size of the spherical region, $r_{s}$, can
be infered from the relationship $\tau = \sigma_0 n r_s$, where
$\sigma_0=5.898\times 10^{-14}$cm$^{-2}$ is the Lyman$\alpha$
cross section at the  line's center, $n$ is the number density. 
This expresion can be rewritten as
\begin{equation}
r_{s} = 0.055 \left(\frac{\tau}{10^7}\right) \left(
\frac{\mathrm{atoms\ cm^{-3}}}{n}\right) \mathrm{kpc}.
\end{equation}
%
Using this result and assuming a bound for the number density of
$10^{-2}<n/\mathrm{atoms\ cm^{-3}}<1$ we have $5.5\times 10^{-2}
< r_s/\mathrm{kpc}<5.5$.

This is consistent with the  constraint on the total HI mass and the
optical depth $\tau_0$ as we show next. The total HI mass can be
approximated as $M_H = \frac{4\pi}{3} m_H n r_s^3$, where
$m_H=1.67\times 10^{-24}$ gr is the mass of a single Hydrogen atom.
This allows us to write the total hydrogen mass as
\begin{equation}
M_{H} = 5.70 \times 10^{6} \left(\frac{\tau}{10^{7}}\right)
\left(\frac{r_s}{\mathrm{kpc}}\right)^2 M_{\odot}
\end{equation}
%
The uppper limit from the HI mass non-detection imposes $r_{s} <
6.81$ kpc, which is consistent with the bound we found in the first
place, as we wanted to show.

Using those size bounds for the spherical model we find a constrain for
the dynamical mass of $5.45\times 10^{8}M_{\odot} < M_{\rm dyn} <
5.45\times 10^{10} M_{\odot}$. 

\end{document}

