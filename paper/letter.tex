\documentclass[a4paper, usenatbib, 12pt]{article}
\usepackage{subfig}
\usepackage{float}
\usepackage{wrapfig}
\usepackage{graphicx}
\usepackage{amsmath}
\usepackage{amssymb}
\usepackage{booktabs}
\usepackage{cite}
\usepackage[icelandic,spanish,english]{babel}
\usepackage[T1]{fontenc}
\usepackage[utf8]{inputenc}
\usepackage[top=3.5cm, bottom=2.5cm, left=3.5cm,right=3.5cm]{geometry} 

%----------------------New commands -------------------

\newcommand{\tol}{Tololo 1214-277}
\newcommand{\lya}{Ly$\alpha$}
\newcommand{\hb}{H$\beta$}
\newcommand{\ha}{H$\alpha$}
\newcommand{\oiii}{[OIII]}
\newcommand{\oii}{[OII]}
\newcommand{\nii}{[NII]}
\newcommand{\esca}{erg cm$^{-2}$ s$^{-1}$ \AA$^{-1}$}
\newcommand{\esc}{erg cm$^{-2}$ s$^{-1}$}
\newcommand{\es}{erg s$^{-1}$}
\newcommand{\esa}{erg s$^{-1}$}
\newcommand{\kms}{km s$^{-1}$}
\newcommand{\apj}{ApJ}  
\newcommand{\jcap}{JCAP}  
\newcommand{\apjs}{ApJS}  
\newcommand{\pasa}{PASA}  
\newcommand{\apjl}{ApJL}  
\newcommand{\aj}{AJ}  
\newcommand{\mnras}{MNRAS}  
\newcommand{\mnrassub}{MNRAS accepted}  
\newcommand{\aap}{A\&A}  
\newcommand{\aaps}{A\&AS}  
\newcommand{\araa}{ARA\&A}  
\newcommand{\nat}{Nature}  
\newcommand{\physrep}{PhR}
\newcommand{\pasp}{PASP}    
\newcommand{\pasj}{PASJ}    
\newcommand{\sigmaclump}{$54.3\pm 0.6$ km s$^{-1}$}
\newcommand{\inftyclump}{$54.3\pm 5.1$ km s$^{-1}$}
\newcommand{\probaclump}{$0.96\pm 0.01$}
\def\simgt{\lower.5ex\hbox{\gtsima}}
\def\simlt{\lower.5ex\hbox{\ltsima}}
%------------------------------------------------------

\begin{document}
\pagestyle{empty}
\noindent
\textbf{Lyman-$\alpha$ Line Sketches a Supermassive Dwarf Galaxy}
\\
\\
Jaime E. Forero-Romero$^{1}$, Max Groenke$^2$, Mar\'ia Camila
Remolina-Guti\'errez$^1$, Juan Nicol\'as
Garavito-Carmargo$^3$, Mark Dijkstra$^2$.
\\
\\
\scriptsize
{$^1$ Departamento de Física, Universidad de los Andes, Cra. 1
  No. 18A-10 Edificio Ip, CP 111711, Bogot\'a, Colombia 
\\
$^2$ Institute of Theoretical Astrophysics, University of Oslo,
Postboks 1029 Blindern, NO-0315 Oslo, Norway.
\\
$^3$ Department of Astronomy, University of Arizona, 933 North Cherry
Avenue, Tucson, AZ 85721, USA. 
\normalsize
\\
\\
\textbf{
  Star-forming Compact Dwarf Galaxies (CDGs) resemble the expected
  pristine conditions of the first galaxies in the Universe.    
Before the observational detection of the first galaxies becomes
reality, CDGs are the best systems to test our ideas on primordial
galaxy formation and evolution.    
Here we report on one of such CDGs, \tol, which presents
a broad symmetric Lyman-$\alpha$ line emission that had evaded theoretical
interpretation so far. 
In this letter we explain these features by two different models: an
homogeneous gaseous sphere undergoing bulk rotation and an interstellar
medium composed by outflowing clumps with additional random motions.
It is the first time that an observed \lya\ spectrum can be explained
assuming either of these physical conditions.
We find that both models independently require high velocities
(either a bulk rotation of $348^{+75}_{-48}$ km s$^{-1}$ or a clump velocity
dispersion of \sigmaclump with outflows of
\inftyclump) consistent with a dynamical mass of at
least a billion solar masses, $6$ times larger than its baryonic mass.   
We argue that a possible explanation for this excess of
dynamical mass is the presence of a supermassive black hole at the
center of \tol. 
This work demonstrates the importance of considering multiphase
physics and rotation among the possible conditions shaping the
\lya\ spectra of the first galaxies.  
Additionally, if future kinematic maps of \tol\ confirm the high
velocities postulated in our model, it would provide new
evidence for dwarf galaxies as hosts of supermassive black
holes.  
}  



The first generation of galaxies trace our cosmic origins. 
They were the first steps in the evolution of galaxies such as the Milky
Way. 
In the standard Big Bang cosmology the only chemical elements that
were created in the nucleosynthesis process were Hydrogen, Helium and
Lithium.  
Heavier elements must have been created in stellar evolution process. 
Therefore, we expect the first generation of
galaxies to be metal free and rich in Hydrogen. 
This kind of primordial galaxies have not been detected yet. 
However, dwarf star forming galaxies with a low metallicity content
are seen as templates to understand the early galaxy evolution process. 

Almost fifty years ago \cite{PartridgePeebles} it was realized that
young galaxies could be detected through a strong Lyman-$\alpha$ line
emission.  
This theoretical prediction was only confirmed thirty years later on
distant, relatively young, not primordial, galaxies \cite{1998ApJ...498L..93D}.
Currently Lyman Alpha Emitting (LAE) galaxies are commonly targeted
in surveys. 
The presence of the Ly-$\alpha$ emission line provides confirmation of
the distance of a galaxy while provides clues about the stellar
population and inter-stellar medium conditions regulating the
Ly-$\alpha$ emission. 

The Ly-$\alpha$ emission line is not exclusive of distant galaxies. 
Any galaxy with low dust content and ongoing star formation has the
right conditions to show this line.  
There are, for instance,  local Universe surveys that target
Ly-$\alpha$ emission in nearby dwarf star forming galaxies 
\cite{LARS}. 
The study of nearby LAE samples has allowed the study of other
indicators that might be more difficult to obtain for distant galaxies
such as morphology, dust attenuation, neutral hydrogen contents and
ionization state.  

However, the physical interpretation of Ly-$\alpha$ observations is
not straightforward \cite{2015ApJ...805...14R}. 
This is due to the resonant nature of the Ly-$\alpha$ line. 
A Ly-$\alpha$ photon follows a diffusion-like process before escaping
the galaxy or being absorbed by dust. 
The resulting line profile becomes sensitive to the dynamical, chemical
and thermal conditions in the interstellar medium. 
There are very few analytically tools available to interpret the
emerging Ly-$\alpha$ line.
They are applicable only in very few cases of highly symmetrical
conditions, which are hardly met in real astrophysical systems.
For these reasons the interpretation of Ly-$\alpha$ observations
requires state-of-the-art Monte Carlo radiative transfer simulations.   


\tol\ is a compact star forming dwarf galaxy that presents a
strong Ly-$\alpha$ emission \cite{Thuan97} with two puzzling 
features: the line is symmetric and single peaked.
Usually the Ly-$\alpha$ line has an asymmetric single or double peak. 
These two special features in \tol\ cannot be explained with
conventional models \cite{2006A&A...460..397V,2015ApJ...812..123G}.  

In this letter we show how the \tol's \lya\ profile can be explained
either by rotation\cite{GaravitoCamargo2014}or the recently developed
class of more complex 
multiphase models that predict a wider variety of spectra
including, single, double and triply peaked spectra \cite{Gronke2016}.  
Figure 1. summarizes our findings.
Dots represent the observational data for \tol\ with the
overplot from our best fits from the analytical solution for a
rotating homogeneous gas sphere (thin line) and the multiphase
model (thick line). 
This is the first time that these models have been introduced with
success to explain an observed \lya\ profile.   


The best parameters in the rotation model are a rotational velocity of 
$V_{\rm max}=348^{+75}_{-48}$ \kms, a neutral Hydrogen optical depth of 
$\log_{10}\tau=6.96^{+0.26}_{-0.18}$,  and an inter-stellar medium temperature of $\log_{10} T/\mathrm {K} = 4.27^{+0.11}_{-0.18}$.  
This model is also able to constrain the angle between the plane
perpendicular to the rotation axis and the observational line-of-sight
to $\theta = 35.78^{+2.13}_{-1.88}$ degrees.

In the multiphase model the best constrained parameters by the
observational data are the clump velocity dispersion
$\sigma_{\rm{cl}}=$\sigmaclump ,  the clump's outflowing velocity
$v_{\infty, \rm{cl}}=$\inftyclump\ and the fraction of the
\lya\ emission that is  coming from the cold clumps  $P_{\rm cl}=$\probaclump.


Assuming that the clumps are located in a spherical region of radius
$r_s=2.25$ kpc (corresponding to \tol's estimated 3D half-light
radius), 
this corresponds to dynamical masses of $M_{\rm dyn} =
3.2_{-1.0}^{+1.6}\times 10^{10} M_{\odot}$ and $M_{\rm
  dyn}=2.31\pm0.04 \times 10^{9}$ $M_{\odot}$ for the
rotation and multiphase models, respectively, 

\tol's stellar mass is  $M_{\star} = 1.45\pm0.45\times 10^{8}
M_{\odot}$   \cite{2014PASP..126.1079M} and its total neutral HI mass is $M_{\rm HI}<2.65\times 10^{8}$ M$_{\odot}$
\cite{pustilnikmartin07}; the dynamical mass is at least 6 to 80 times
the baryonic mass, depending if one considers the multiphase or
rotation estimate. 

We lean towards the lower dynamical mass estimate from the multiphase
model as it seems easier to reconcile with the following two
astrophysical mechanisms for its origin.
The first way to explain a dynamical mass of $10^{9} M_{\odot}$
in a sphere of 2 kpc could be having a dark matter halo of at
least $10^{12}$ M$_{\odot}$ in mass \cite{2011ApJ...726..108T}, which
leaves open the question as to why  \tol\ is not more similar to the
Milky Way galaxy as it is hosted by a dark matter halo of similar
mass. 
A second possibility is that \tol\ hosts a supermassive black hole of
$10^{9} M_{\odot}$. This is almost two orders of magnitude higher
than the supermassive black hole found in the compact dwarf galaxy
M60-UCD1 \cite{2014Natur.513..398S}, which has a similar stellar mass
as \tol. This would leave open the question about the formation
process of such a system.   
 

Another perspective to appreciate the atypically high dynamical mass
estimates comes from the observed scaling relations for dwarf
galaxies.
Assuming that \tol\ followed the fundamental plane relationship
between its mean surface brightness $I_e$, the projected half-light
radius $R_e$ and the velocity dispersion $\sigma$, described by $\log
I_e=1.6 \log\sigma - 1.21\log R_e + 0.55$ \cite{2009ApJ...698.1590G},
the expected velocity dispersion should be on the order of $5 \pm 1$ 
\kms, which is a factor of $\sim$ $10$ - $60$ lower than the 
results from the multiphase and rotation models, respectively. 
These are equivalent to factors of $\sim$ $100$ - $3600$ on the
dynamical mass.
Once again,  \tol\ seems to be significantly more massive than
expected. 




A new observational test is needed to clarify the physical nature of
\tol. 
We suggest that integral field unit measurements spatially
resolving its spatial extent are up to the task. 
\tol\ spans a region of $4$ arcseconds,
an instrument such as the Multi Unit Spectroscopic Explorer
\cite{2014Msngr.157...13B} with its
nominal $0.2$ arcseconds spatial sampling over a $1.0$ arcminute field
in wide-field mode could provide a coarse mapping of different
ionization lines to infer a kinematic map.
Another observational test includes the measurement of the
\lya\ ionizing continuum escape fraction.
In the rotational model this fraction should be zero, while
the multiphase model predicts that averaging over all sightlines
it should be around $0.5^{+1.0}_{-0.4}$\%, with the possibility of strong
variations depending on viewing angle. 

All in all, the mere existence of a strong LAE galaxy with a broad,
symmetric line is interesting.
It raises the question whether some high redshift LAEs have asymmetric
lines because the blue half was truncated by the intergalactic medium.
In this case the \lya\ radiation could emerge as a low surface
brightness glow, which may be connected to \lya\ halos, while also
influencing the way LAEs can be used as a probe of reionization
\cite{2014PASA...31...40D}.  

These findings demonstrate the importance of including rotation and multiphase
conditions as features to model the \lya\ line in high redshift
galaxies.
Additionally, if the hypothesis of a supermassive black
hole in \tol\ proves to be consistent with future observational
kinematic maps, it could correspond to a so far undetected
supermassive black hole in a dwarf galaxy, providing a new way to test
and probe theories on the co-evolution of galaxies and black holes in
the first generation of galaxies.   

\begin{figure}
\begin{center}
\includegraphics[width=0.8\textwidth]{CLARA-TOL-main.pdf}
\caption{{\bf Broad, single peaked and symmetric Ly-$\alpha$ emission of \tol.}
  Dots correspond to the observational data. The line shows the results
of our best model from a full radiative transfer simulation both for
the rotation and multiphase models.} 
\end{center}
\end{figure}

\bibliography{references}{}
\bibliographystyle{ieeetr}

\newpage 

\section*{\tol\ characteristics}


\begin{table}
\begin{center}
\begin{tabular}{lc}
$\alpha$(2000)$^{a}$ & 12h17min17.1s\\
$\delta$(2000)$^{b}$ & -28d02m32s\\
$l$, $b$ (deg) & 294, 34\\
$m_V$ & 17.5\\
  M$_V$ & -17.6\\ 
$v$(km s$^{-1}$) & 7795\\
Ly-$\alpha$ (erg cm$^{-2}$ s$^{-1}$ \AA$^{-1}$)& $8.1\times 10^{-14}$ \\
Ly-$\alpha$ EW & $70$\AA\\
H$\beta$ (erg cm$^{-2}$ s$^{-1}$ \AA$^{-1}$) & $1.62\times 10^{-14}$ \\
$21$cm (Jy km s$^{-1}$)& $<0.10$ \\
\end{tabular}
\end{center}
\caption{Basic observational characteristics of TOL1214-277
  \cite{Thuan97}\\} 
\end{table}


\tol\ receding velocity is $7785\pm 50$km s$^{-1}$, which translates
into a distance of $106.6$ Mpc (with the Hubble constant $H_{0}$=73
Mpc km s$^{-1}$) 
Its metallicity is $\sim Z_{\odot}/24$ \cite{Izotov04} as derived from optical
spectroscopy. 


The observed flux for the Lyman alpha line is $\sim
8.1\times 10^{-14}$ erg cm$^{-2}$ s$^{-1}$ \cite{Thuan97}
and a Equivalent Width of $70$\AA and its H$\beta$ flux is 
$1.62\times 10^{-14}$ erg cm$^{-2}$ s$^{-1}$ \AA${-1}$
\cite{Izotov04} which gives a Ly$\alpha$/H$\beta$ flux ratio of
4.9$\pm$0.1. The Ly-$\alpha$ flux values correspond to luminosities of
$L_{Ly\alpha}=2.2\times 10^{42}$ erg s$^{-1}$ over a $20$\AA
bandwidth, which in turns translates  into a star formation rate of
$2.0$ M$_{\odot}$ yr$^{-1}$ using a standard conversion factor between
luminosity and star formation rate of $9.1\times 10^{-43}$
$L_{Ly\alpha}$ M$_{\odot}$ yr$^{-1}$. 
The absolute magnitude in the $V$ band translates into a luminosity of
$8.9\times 10^{8}$ L$_{\odot}$.
% using http://tomdwelly.com/tools_fluxtolum.php
Comparing this ratio with the theoretical expectation from case B
recombination of $23.3$ \cite{Hummer1987} one can estimate an escape
fraction of $20$\% for Ly$\alpha$ radiation.
The bolometric UV luminosity is $9.43\pm1.94 \times 10^{8}$
L$_{\odot}$ as measured by GALEX. 


There is an upper limit for the  
integrated flux of $<0.10$ Jy km s$^{-1}$, which translates into a
upper limit for the HI mass of $M<2.65\times 10^{8}$ M$_{\odot}$
\cite{pustilnikmartin07}.  

The near-infrared fluxes at $3.6$ $\mu$m and $4.5$ $\mu$m are
$7.71\pm0.55\times 10^{-5}$ Jy and $7.98\pm0.71\times 10^{-5}$ Jy
\cite{2008ApJ...678..804E}.  Using a conversion between fluxes and
stellar mass calibrated on the Large Magellanic Cloud $M_{\star} =
10^{5.65} \times F_{3.6}^{2.85} \times F_{4.5}^{-1.85} \times
(D/0.05)^2 M_{\odot}$, where fluxes are in Jy and $D$ is the luminosity
distance to the source in Mpc, we find $M_{\star} = 1.45\pm0.45\times 10^{8}
M_{\odot}$, with a $30\%$ uncertainty coming from the calibration
process \cite{2012AJ....143..139E}.  


We computed the projected half-luminosity radius to be $R_s=1.5\pm0.1$ kpc 
from the surface intensity profiles reported by.
\cite{2003A&A...410..481N}. 
Assuming spherical geometry, one can translate this value into a 3D
half-luminosity radius of $r_s=3R_s/2=2.25$ kpc.



\section*{The Rotation Model}

The rotation model corresponds to the work presented in
\cite{GaravitoCamargo2014} based on the Monte Carlo code
\texttt{CLARA} \cite{CLARA}. 
In that model the Ly-$\alpha$ photons are propagated 
within a spherical and homogeneous cloud of HI gas undergoing solid
body rotation.
The sphere is fully characterized by three parameters: the HI optical
depth $\tau$ measured from the center to its surface, the HI
temperature $T$, and the linear surface velocity $V_{\rm max}$.  
Photons are emitted at their natural frequency from the center of the
sphere. 
Including the effect of dust only changes the overall line
normalization but not its shape.  
The results we report in the main body of the paper do not include any
dust model.
In this letter we use an analytical solution that captures the most
important effects of rotation onto the \lya\ line.



The first important effect of rotation is that it breaks the spherical
symmetry of the static case. 
Now the line's observed morphology depends on the angle $\theta$ between the
line-of-sight (LOS) and the rotation axis. 
LOS parallel to the rotation axis tend to observed the line without
any modification from rotation, while the perpendicular LOS will
observe a maximal change in the line's morphology due to rotation.

The main change in the line's morphology is that it broadens and the
intensity at the center increases. 
For high enough rotational velocities the intensity at the peak's
center increases so much that the line goes from double to single
peaked, sometimes slightly triple peaked.
This is the feature that allows this model to fit the observational
features of \tol.

\cite{GaravitoCamargo2014} derive a concise analytical description for
those features. 
This description takes into account how different parts of the
sphere's surface shift in frequency the \lya\ photons. 
Different shifts in frequency come from different values for the projected
velocity along the LOS. 
Using the analytical solution for the \lya\ spectra of a static sphere
plus the right frequency shifts  computed from geometrical
considerations, one is able to compute an analytical solution for the
rotating sphere that reproduces the main features found using the full
numerical simulation. 

The analytical solution for the rotation sphere was the base to
perform the Markov Chain Monte Carlo (MCMC) calculation using the
\texttt{emcee} Python library \cite{2013PASP..125..306F}. \texttt{emcee} 
is an open source optimized implementation of the affine-invariant 
ensemble sampler for MCMC. The algorithm creates a number of walkers that,
during a sufficient number of steps, generate parameters' combinations for
a specific model. For each time, the code calculates the likelihood of the
combination with respect to the observational data. The walkers ecplore
the parameter space sampling the likelihood function.

MCMC methods are optimal for sampling parameters at a high number of 
dimensions. In this case we explore flat priors on four parameters:
$200<V_{\rm max}/\mathrm{km\ s}^{-1} <600$,  
$6.0<\log_{10}\tau<9.0$, $4.0<\log_{10} T/10^4\mathrm{K}< 4.5$ and
$0<\theta<90$ using $500$ steps with $24$ walkers for a total of
$12000$ points in the chain. 
The results are summarized in 
Figure \ref{emceeresults}. 
From this model we find that the fiducial 
parameters that could explain the broad features in \tol\ are 
$V_{\rm max}=348^{+75}_{-48}$ km s$^{-1}$, $\log \tau = 6.96^{+0.26}_{-0.18}$, 
$\log_{10} T/\mathrm {K} = 4.27^{+0.11}_{-0.18}$ and $\theta = 35.78^{+2.13}_{-1.88}$ 
degrees.


\begin{figure}
\begin{center}
\includegraphics[width=1.0\textwidth]{emcee_results.pdf}
\caption{{\bf Results from the Markov Chain Monte Carlo computation for
    the rotation model.} The dotted vertical lines in the outer histograms 
	represent the percentiles 16\%, 50\% and 84\%. \label{emceeresults}} 
\end{center}
\end{figure}




%In the rotation model assuming reasonable bounds for the number
%density of neutral Hydrogen atoms (
%$0.01<n/\mathrm{atoms\ cm^{-3}}<0.1$) and using $\tau=10^7$ the radius
%of the emission region can be bracketed to be in $0.55 < r_s/\mathrm{kpc}< 5.5$.

\section*{The Multiphase Model} 

The idealized multiphase model consists of spherical, cold, dens
clumps of neutral hydrogen (and dust) embedded in a hot, ionized
medium. 
The clumps also have a random and an outflowing velocity
component which totals the number of parameters describing the model
to be $14$. 

In order to map out this large parameter space, we randomly drew
$2500$ sets of parameters within a observationally realistic range
(based on the considerations of \cite{Laursen2013ApJ...766..124L})
yielding a large variety of single-, double- and triple-peaked
spectra. 
The full analysis of the the spectral features as well as
more details on the radiative transfer are presented in
\cite{Gronke2016}.    
 For the current work, we computed the $\chi^2$ for each of the $2500$
models. We selected the best $50$ models with the lowest $\chi^2$. The
$\chi^2$ gap in those $50$ models is close to $3000$, the lowest
$\chi^2$ is close to $1200$. The total number of degrees of freedom is
$104$.    

Then we performed a Kolmogorov-Smirnov test to compare each parameter
distribution in the best $50$ models against the parent distribution
of $2500$ models. 
If we obtain a p-value $<0.05$ for a given parameter, we conclude that
this parameter does influence the $\chi^2$ fit, as the distribution for
the best $\chi^2$ models is statistically different to the
distribution from the global sample of $2500$ models.  

From this test we found that only three parameters influence the
$\chi^2$: the clump outflow velocity $v_{\infty,\rm{cl}}$ (p-value 
$10^{-18}$), the clump velocity dispersion $\sigma_{\rm cl}$ (p-value
$10^{-4}$) and the probability that the \lya\ emission comes from the
clumps $P_{\rm cl}$ (p-value $10^{-4}$).

The best values for those parameters that we report here correspond to
the values that produce the minimum
$\chi^2$. The 1-$\sigma$ uncertainty comes from a parabolic fit to the
$\chi^2$ as a function of $v_{\infty,\rm{cl}}$, $\sigma_{\rm cl}$,
$P_{\rm cl}$ around its corresponding minimum.
Under these conditions we find $\sigma_{\rm cl}=$\sigmaclump,
$v_{\infty {\rm, cl}}=$\inftyclump\ and $P_{\rm cl}=$\probaclump. 

Qualitatively as \tol\ possesses a very wide spectrum which can be
achieved by subsequent scatterings off (relatively) fast moving clumps
while the multi-phase nature (i.e., the existence of low-density
channels) ensures the high flux at line center as observed.     


\section*{Dynamical Mass Estimates}

Having constrains for  velocity dipersion $\sigma$ of some dynamical
tracers (clumps in the case of the multiphase model) in a spherical
system located in a region of size $r$ we estimate the dynamical mass
within $r$. 


\begin{equation}
M_{\rm dyn} = 3 \frac{\sigma ^{2}r}{G} = 3.48\times10^{9}
\left(\frac{\sigma}{100\ \mathrm{km\ s}^{-1}}\right)^2\left(\frac{r}{\mathrm{kpc}}\right) M_{\odot}
\end{equation}

As described in the main the letter, we use the 3D half-luminosity
radius, $r_{s}$, as the typical size for the HI region. 

In the case of rotational velocity $v$ in a region of size $r$ we
estimate the dynamical mass by 

\begin{equation}
M_{\rm dyn} = \frac{v ^{2}r}{G} = 1.16\times10^{9}
\left(\frac{v}{100\ \mathrm{km\ s}^{-1}}\right)^2\left(\frac{r}{\mathrm{kpc}}\right)M_{\odot} 
\end{equation}

%From the multiphase model we obtained $v=$\sigmaclump
%and $r=5$ kpc. This corresponds to a dynamical mass of $M_{\rm
%  dyn}=2.8^{+1.3}_{-1.2}\times 10^{9}$ $M_{\odot}$ which is at least
%$5$ times the HI mass plus the stellar mass estimated from
%observations.   

%In the rotation model the size of the spherical region, $r_{s}$, can
%be inferred from the relationship $\tau = \sigma_0 n r_s$, where
%$\sigma_0=5.898\times 10^{-14}$cm$^{-2}$ is the Lyman$\alpha$
%cross section at the  line's center, $n$ is the number density. 
%This expression can be rewritten as
%\begin{equation}
%r_{s} = 0.055 \left(\frac{\tau}{10^7}\right) \left(
%\frac{\mathrm{atoms\ cm^{-3}}}{n}\right) \mathrm{kpc}.
%\end{equation}
%
%Using this result and assuming a bound for the number density of
%$0.01<n/\mathrm{atoms\ cm^{-3}}<0.1$ we have $0.55
%< r_s/\mathrm{kpc}<5.5$.

%This is consistent with the  constraint on the total HI mass and the
%optical depth $\tau_0$ as we show next. The total HI mass can be
%approximated as $M_H = \frac{4\pi}{3} m_H n r_s^3$, where
%$m_H=1.67\times 10^{-24}$ gr is the mass of a single Hydrogen atom.
%This allows us to write the total hydrogen mass as
%\begin{equation}
%M_{H} = 5.70 \times 10^{6} \left(\frac{\tau}{10^{7}}\right)
%\left(\frac{r_s}{\mathrm{kpc}}\right)^2 M_{\odot}
%\end{equation}
%
%The upper limit from the HI mass non-detection imposes $r_{s} <
%6.81$ kpc, which is consistent with the bound we found in the first
%place, as we wanted to show.

%Using those size bounds for the spherical model we find a constrain for
%the dynamical mass of $5.45\times 10^{9}M_{\odot} < M_{\rm dyn} <
%5.45\times 10^{10} M_{\odot}$. 

\end{document}

